\subsection{}

\begin{tcolorbox}
    Welches Flächenträgheitsmoment hat ein runder Stab mit Radius $r$?
\end{tcolorbox}

\noindent Auch hier nutzen wir wieder die Definition aus \autoref{eq:def-flaechentraegheitsmoment}.

\noindent Hier nehmen wir für den Kreis aber Zylinderkoordinaten.

Durch die Transformation ($y^{2} \cdot dxdy \to r^{3} \cdot d\varphi dr$) lautet das Integral dann:

\begin{align*}
    I &= \int_{0}^{2 \pi} d\varphi \int_{0}^{r} dr' ~ r'^{3} \\
    &= \varphi \int_{0}^{r} dr' ~ r'^{3} \bigg|_{\varphi = 0}^{\varphi = 2\pi} \\
    &= 2\pi \frac{1}{4} r'^{4} \bigg|_{r' = 0}^{r'=r} \\
    &= \frac{1}{2}\pi r^{4} \\
\end{align*}

\begin{tcolorbox}[result]
    Somit lautet das Flächenträgheitsmoment eines Kreisstabes:
    \begin{align*}
    I = \frac{1}{4} r^{4} \pi
    \end{align*}
\end{tcolorbox}

